\documentclass{letter}
\usepackage[top=0.5cm, bottom=0.5cm]{geometry}
\usepackage{xcolor}
\date{\today}

% \newcommand{\master}{M.Sc. Mathematics}
% \newcommand{\uni}{University of Bonn}
% \newcommand{\city}{Bonn}
% \newcommand{\scholar}{DAAD grant}

\newcommand{\program}{Ph.D.\ position}
\newcommand{\uni}{Imperial college of London}
\newcommand{\city}{UK}
% \newcommand{\scholar}{PSL grant}

%% for scholarship 
\begin{document}
\begin{letter}
  \hfill

  \textbf{Research Proposal: Aerodynamics Investigation for Enhanced Performance in Boeing Aircraft}

  Aerodynamics is a critical factor influencing the overall performance and efficiency of aircraft. This research proposal seeks to explore the complexities of aerodynamics, specifically targeting Boeing aircraft, with the  aim of improving their performance, fuel efficiency, and environmental sustainability.

  The primary goal of this 3-year research project is to conduct a thorough investigation into the aerodynamic performance of an aircraft wing, using advanced computational fluid dynamics (CFD) simulations. These simulations will provide a detailed understanding of the airflow around the wing, allowing us to identify potential areas for optimization and enhancement.

  This research will explore how improvements in aerodynamics can positively impact overall aircraft performance, including factors such as speed, stability, and fuel consumption. By gaining insights into the intricate aerodynamic characteristics of Boeing aircraft, we aim to propose practical and feasible recommendations for achieving enhanced efficiency and environmental sustainability in aviation.

  This research endeavor aligns with the broader industry goal of developing more eco-friendly and technologically advanced aircraft. The outcomes of this study have the potential to influence not only the design and manufacturing processes of Boeing aircraft but also contribute valuable insights to the aviation sector as a whole.

  \bigskip

  Víctor Ballester
\end{letter}
\end{document}
